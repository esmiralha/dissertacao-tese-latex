%A classe é a dcc-nce, e o parâmetro a ser informado é diss (para dissertação de mestrado)
\documentclass[diss]{dcc-nce}
\usepackage[T1]{fontenc}
\usepackage{color,graphicx}
\usepackage{graphics}
\usepackage{url}
\usepackage{amsmath,amssymb}

%Para citar textualmente ao invés de com números, basta
% escrever \usepackage{natbib} ao invés da linha abaixo.
\usepackage{natbib}

\usepackage{dsfont} %Usado para conjuntos N, Z, Q, R, C
\usepackage[portuguese,algoruled,longend]{algorithm2e}
\usepackage{algorithmic}
\usepackage[utf8]{inputenc}
\usepackage{listings}%Para inserir codigos fontes de programas no apendice.
\usepackage{xcolor}
% Definindo novas cores
\definecolor{verde}{rgb}{0,0.5,0}
% Configurando layout para mostrar codigos C++
\usepackage{listings}
\lstset{
  language=C++,
  basicstyle=\ttfamily\small,
  keywordstyle=\color{blue},
  stringstyle=\color{verde},
  commentstyle=\color{red},
  extendedchars=true,
  showspaces=false,
  showstringspaces=false,
  numbers=left,
  numberstyle=\tiny,
  breaklines=true,
  backgroundcolor=\color{green!10},
  breakautoindent=true,
  captionpos=b,
  xleftmargin=0pt,
}

% Para contagem do numero total de folhas:
\usepackage{everyshi}
\makeatletter
\let\totalpages\relax
\newcounter{mypage}
\EveryShipout{\stepcounter{mypage}}
\AtEndDocument{\clearpage
   \immediate\write\@auxout{%
    \string\gdef\string\totalpages{\themypage}}}
\makeatother


\topmargin=0in
\textheight=20.5cm


\begin{document}

%Este tem que vir primeiro neste arquivo, caso contrario nao aparecerao
%as palavras-chave na ficha catalografica:
\input{pre-text/palavrasChavePortugues}    % Editar o arquivo palavrasChavePortugues

%O restante vem depois:
\title{Detecção automática de denúncias de crime fraudulentas através de análise do discurso}

% COLOCAR AQUI O CUTTER (CÓDIGO FORNECIDO PELA BIBLIOTECA DO NCE PARA A FICHA CATALOGRÁFICA):
\codigobiblioteca{CBIB}

\author{Esmiralha}{Luiz Barreto de Castro}
\advisor[Prof.~Dr.~]{Último Sobrenome}{Nome Sobrenome1 Sobrenome2 ...}
\coadvisor[Profa. Dra.]{Último Sobrenome}{Nome Sobrenome1 Sobrenome2 ...}
\banca{Profa. Dr. Fulano de Tal}{Prof. Dr. Beltrano}{Prof. Dr. Ciclano}
\date{2019}
\maketitle
                      % Editar o arquivo capa.tex

\pagenumbering{roman} %numeração de páginas em romano começa a partir do primeiro capítulo

\input{pre-text/dedicatoria}     % Editar o arquivo dedicatoria.tex
\input{pre-text/agradecimentos}  % Editar o arquivo agradecimentos.tex
\input{pre-text/resumoPortugues} % Editar o arquivo resumoPortugues.tex
%\begin{englishabstract}{Colocar aqui o título da dissertação em inglês}{coloque aqui as palavras-chave em inglês, separadas por vírgula}

\begin{englishabstract}{}{deception detection,natural language processing,named entity recognition}


%MADUREIRA, Rodrigo Lopes Rangel. Algoritmos de interseções, bla bla bla...\\

Colocar aqui o resumo em inglês.
\end{englishabstract}

    % Editar o arquivo resumoIngles.tex

\listoffigures{}

\listoftables{}

%\input{pre-text/listaAbreviaturaSiglas}    % Editar o arquivo listaAbreviaturaSiglas.tex

\tableofcontents{}                  % Sumário

\parindent=1.25cm %start for each paragraph from the left margin
\parskip=20pt
\baselineskip=20pt

\pagenumbering{arabic} %numeração de páginas em arábico começa a partir do primeiro capítulo

\input{chapters/introducao}	% Editar o arquivo introducao.tex
\chapter{Nome do Capítulo A}

Este capítulo apresenta um levantamento do estado da arte em armazenamento de documentos XML. Existem várias propostas na literatura blabla...

\begin{table}
\caption{Comparação dos trabalhos\label{T1}} %título de tabelas sempre aparecem antes da tabela
\center{
\begin{tabular}{l|l}
\hline
Titulo Coluna 1   & Título Coluna 2\\
\hline
X                 & Y\\
X                 & W\\
\hline
\end{tabular}}
\end{table}

Os trabalhos blablabla são comparados na Tabela \ref{T1}. 
	% Editar o arquivo capituloA.tex
\input{chapters/capituloB}	% Editar o arquivo capituloB.tex
\input{chapters/conclusao}	% Editar o arquivo introducao.tex

\bibliography{post-text/referencias}       % Editar o arquivo "referencias.bib"
\bibliographystyle{latex-stuff/abnt-ufrgs} % Procura pelo arquivo "abnt-ufrgs" - normas ABNT.

\clearpage

\appendix
\input{post-text/apendice}          % Editar o arquivo apendice.tex

\end{document}
